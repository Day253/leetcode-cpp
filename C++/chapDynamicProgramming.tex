\chapter{动态规划}


\section{Maximum Subarray} %%%%%%%%%%%%%%%%%%%%%%%%%%%%%%
\label{sec:maximum-subarray}


\subsubsection{描述}
Find the contiguous subarray within an array (containing at least one number) which has the largest sum.

For example, given the array \code{[−2,1,−3,4,−1,2,1,−5,4]},
the contiguous subarray \code{[4,−1,2,1]} has the largest \code{sum = 6}.


\subsubsection{分析}
最大连续子序列和,非常经典的题。

当我们从头到尾遍历这个数组的时候,对于数组里的一个整数,它有几种选择呢?它只有两种选择: 1、加入之前的SubArray;2. 自己另起一个SubArray。那什么时候会出现这两种情况呢?

如果之前SubArray的总体和大于0的话,我们认为其对后续结果是有贡献的。这种情况下我们选择加入之前的SubArray

如果之前SubArray的总体和为0或者小于0的话,我们认为其对后续结果是没有贡献,甚至是有害的(小于0时)。这种情况下我们选择以这个数字开始,另起一个SubArray。

设状态为d[j],表示以S[j]结尾的最大连续子序列和,则状态转移方程如下:
\begin{eqnarray}
d[j] &=& \max\left\{d[j-1]+S[j],S[j]\right\}, \text{ 其中 }1 \leq j \leq n \nonumber \\
target &=& \max\left\{d[j]\right\}, \text{ 其中 }1 \leq j \leq n \nonumber
\end{eqnarray}

解释如下:
\begindot
\item 情况一,S[j]不独立,与前面的某些数组成一个连续子序列,则最大连续子序列和为$d[j-1]+S[j]$。
\item 情况二,S[j]独立划分成为一段,即连续子序列仅包含一个数S[j],则最大连续子序列和为$S[j]$。
\myenddot  

其他思路:
\begindot
\item 思路1:直接在i到j之间暴力枚举,复杂度是$O(n^3)$
\item 思路2:处理后枚举,连续子序列的和等于两个前缀和之差,复杂度$O(n^2)$。
\item 思路3:分治法,把序列分为两段,分别求最大连续子序列和,然后归并,复杂度$O(n\log n)$
\item 思路4:把思路2$O(n^2)$的代码稍作处理,得到$O(n)$的算法
\item 思路5:当成M=1的最大M子段和
\myenddot


\subsubsection{代码}
\begin{Code}
// LeetCode, Maximum Subarray
class Solution {
public:
    int maxSubArray(int A[], int n) {
        return mcss(A, n);
        //return mcss_dp(A, n);
    }
private:
    /**
     * @brief 最大连续子序列和,思路四
     * @param[in] S 数列
     * @param[in] n 数组的长度
     * @return 最大连续子序列和
     */
    static int mcss(int S[], int n) {
        int i, result, cur_min;
        int *sum = (int*) malloc((n + 1) * sizeof(int));  // 前n项和

        sum[0] = 0;
        result = INT_MIN;
        cur_min = sum[0];
        for (i = 1; i <= n; i++) {
            sum[i] = sum[i - 1] + S[i - 1];
        }
        for (i = 1; i <= n; i++) {
            result = max(result, sum[i] - cur_min);
            cur_min = min(cur_min, sum[i]);
        }
        free(sum);
        return result;
    }

    /**
     * @brief 最大连续子序列和,动规
     * @param[in] S 数列
     * @param[in] n 数组的长度
     * @return 最大连续子序列和
     */
    static int mcss_dp(int S[], int n) {
        int i, result;
        int *d = (int*) malloc(n * sizeof(int));
        d[0] = S[0];
        result = d[0];
        for (i = 1; i < n; i++) {
            d[i] = max(S[i], d[i - 1] + S[i]);
            if (result < d[i])
                result = d[i];
        }
        free(d);
        return result;
    }
};
\end{Code}


\subsubsection{相关题目}
\begindot
\item Binary Tree Maximum Path Sum,见 \S \ref{sec:binary-tree-maximum-path-sum}
\myenddot


\section{Palindrome Partitioning II} %%%%%%%%%%%%%%%%%%%%%%%%%%%%%%
\label{sec:palindrome-partitioning-ii}


\subsubsection{描述}
Given a string s, partition s such that every substring of the partition is a palindrome.

Return the minimum cuts needed for a palindrome partitioning of s.

For example, given \code{s = "aab"},

Return 1 since the palindrome partitioning \code{["aa","b"]} could be produced using 1 cut.


\subsubsection{分析}
定义状态\fn{f(i,j)}表示区间\fn{[i,j]}之间最小的cut数,则状态转移方程为 
$$
f(i,j)=\min\left\{f(i,k)+f(k+1,j)\right\}, i \leq k \leq j, 0 \leq i \leq j<n
$$
这是一个二维函数,实际写代码比较麻烦。
 
所以要转换成一维DP。如果每次,从i往右扫描,每找到一个回文就算一次DP的话,就可以转换为\code{f(i)=区间[i, n-1]之间最小的cut数},n为字符串长度,则状态转移方程为
$$
f(i)=\min\left\{f(j+1)+1\right\}, i \leq j<n
$$

一个问题出现了,就是如何判断\fn{[i,j]}是否是回文?每次都从i到j比较一遍?太浪费了,这里也是一个DP问题。

定义状态\fn{P[i][j] = true if [i,j]为回文},那么
\begin{Code}
P[i][j] = str[i] == str[j] && P[i+1][j-1]
\end{Code}


\subsubsection{代码}
\begin{Code}
//LeetCode, Palindrome Partitioning II
class Solution {
public:
    int minCut(string s) {
        const int len = s.size();
        int f[len+1];
        bool p[len][len];
        //the worst case is cutting by each char
        for (int i = 0; i <= len; i++)
            f[i] = len - 1 - i; // 最后一个f[len]=-1
        for (int i = 0; i < len; i++)
            for (int j = 0; j < len; j++)
                p[i][j] = false;
        for (int i = len - 1; i >= 0; i--) {
            for (int j = i; j < len; j++) {
                if (s[i] == s[j] && (j - i < 2 || p[i + 1][j - 1])) {
                    p[i][j] = true;
                    f[i] = min(f[i], f[j + 1] + 1);
                }
            }
        }
        return f[0];
    }
};
\end{Code}


\subsubsection{相关题目}
\begindot
\item Palindrome Partitioning,见 \S \ref{sec:palindrome-partitioning}
\myenddot


\section{Maximal Rectangle} %%%%%%%%%%%%%%%%%%%%%%%%%%%%%%
\label{sec:maximal-rectangle}


\subsubsection{描述}
Given a 2D binary matrix filled with 0's and 1's, find the largest rectangle containing all ones and return its area.


\subsubsection{分析}
无


\subsubsection{代码}
\begin{Code}
// LeetCode, Maximal Rectangle
class Solution {
public:
    int maximalRectangle(vector<vector<char> > &matrix) {
        if (matrix.empty())  return 0;

        const int m = matrix.size();
        const int n = matrix[0].size();
        vector<int> H(n, 0);
        vector<int> L(n, 0);
        vector<int> R(n, n);

        int ret = 0;
        for (int i = 0; i < m; ++i) {
            int left = 0, right = n;
            // calculate L(i, j) from left to right
            for (int j = 0; j < n; ++j) {
                if (matrix[i][j] == '1') {
                    ++H[j];
                    L[j] = max(L[j], left);
                } else {
                    left = j+1;
                    H[j] = 0; L[j] = 0; R[j] = n;
                }
            }
            // calculate R(i, j) from right to left
            for (int j = n-1; j >= 0; --j) {
                if (matrix[i][j] == '1') {
                    R[j] = min(R[j], right);
                    ret = max(ret, H[j]*(R[j]-L[j]));
                } else {
                    right = j;
                }
            }
        }
        return ret;
    }
};
\end{Code}


\subsubsection{相关题目}

\begindot
\item 无
\myenddot


\section{Best Time to Buy and Sell Stock III} %%%%%%%%%%%%%%%%%%%%%%%%%%%%%%
\label{sec:best-time-to-buy-and-sell-stock-iii}


\subsubsection{描述}
Say you have an array for which the i-th element is the price of a given stock on day i.

Design an algorithm to find the maximum profit. You may complete at most two transactions.

Note: You may not engage in multiple transactions at the same time (ie, you must sell the stock before you buy again).


\subsubsection{分析}
设状态$f(i)$,表示区间$[0,i](0 \leq i \leq n-1)$的最大利润,状态$g(i)$,表示区间$[i, n-1](0 \leq i \leq n-1)$的最大利润,则最终答案为$\max\left\{f(i)+g(i)\right\},0 \leq i \leq n-1$。

允许在一天内买进又卖出,相当于不交易,因为题目的规定是最多两次,而不是一定要两次。

将原数组变成差分数组,本题也可以看做是最大$m$子段和,$m=2$,参考代码:\myurl{https://gist.github.com/soulmachine/5906637}

\subsubsection{代码}
\begin{Code}
// LeetCode, Best Time to Buy and Sell Stock III
class Solution {
public:
    int maxProfit(vector<int>& prices) {
        if (prices.size() < 2) return 0;

        const int n = prices.size();
        vector<int> f(n, 0);
        vector<int> g(n, 0);

        for (int i = 1, valley = prices[0]; i < n; ++i) {
            valley = std::min(valley, prices[i]);
            f[i] = std::max(f[i - 1], prices[i] - valley);
        }

        for (int i = n - 2, peak = prices[n - 1]; i >= 0; --i) {
            peak = std::max(peak, prices[i]);
            g[i] = std::max(g[i], peak - prices[i]);
        }

        int maxProfit = 0;
        for (int i = 0; i < n; ++i)
            maxProfit = max(maxProfit, f[i] + g[i]);

        return maxProfit;
    }
};
\end{Code}


\subsubsection{相关题目}
\begindot
\item Best Time to Buy and Sell Stock,见 \S \ref{sec:best-time-to-buy-and-sell-stock}
\item Best Time to Buy and Sell Stock II,见 \S \ref{sec:best-time-to-buy-and-sell-stock-ii}
\myenddot


\section{Triangle} %%%%%%%%%%%%%%%%%%%%%%%%%%%%%%
\label{sec:triangle}


\subsubsection{描述}
Given a triangle, find the minimum path sum from top to bottom. Each step you may move to adjacent numbers on the row below.

For example, given the following triangle
\begin{Code}
[
     [2],
    [3,4],
   [6,5,7],
  [4,1,8,3]
]
\end{Code}
The minimum path sum from top to bottom is 11 (i.e., 2 + 3 + 5 + 1 = 11).

Note: Bonus point if you are able to do this using only $O(n)$ extra space, where n is the total number of rows in the triangle.


\subsubsection{分析}
设状态为$f(i, j)$,表示从从位置$(i,j)$出发,路径的最小和,则状态转移方程为
$$
f(i,j)=\min\left\{f(i,j+1),f(i+1,j+1)\right\}+(i,j)
$$


\subsubsection{代码}
\begin{Code}
// LeetCode, Triangle
class Solution {
public:
    int minimumTotal (vector<vector<int>>& triangle) {
        for (int i = triangle.size() - 2; i >= 0; --i)
            for (int j = 0; j < i + 1; ++j)
                triangle[i][j] += std::min(triangle[i + 1][j],
                        triangle[i + 1][j + 1]);

        return triangle [0][0];
    }
};
\end{Code}


\subsubsection{相关题目}
\begindot
\item 无
\myenddot