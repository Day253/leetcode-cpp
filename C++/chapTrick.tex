\chapter{编程技巧}

动态分配定长数组,不要用\fn{ary = new type[len]},用std::vector<type> ary(len)。首先在性能上,由于vector能够保证连续内存,因此一旦分配了后,它的性能跟原始数组相当,其次使用方便,用new 还需要多写一句delete,容易出错,且声明多维数组的话,只能一个一个new,例如:
\begin{Code}
int** ary = new int*[row_num];
for(int i = 0; i < row_num; ++i)
    ary[i] = new int[col_num];
\end{Code}
用vector的话一行代码搞定,
\begin{Code}
vector<vector<int> > ary(row_num, vector<int>(col_num, 0));
\end{Code}

在判断两个浮点数a和b是否相等时,不要用\fn{a==b},应该判断二者之差的绝对值\fn{fabs(a-b)}是否小于某个阀值,例如\fn{1e-9}。

