\chapter{字符串}


\section{Valid Palindrome} %%%%%%%%%%%%%%%%%%%%%%%%%%%%%%
\label{sec:valid-palindrome}


\subsubsection{描述}
Given a string, determine if it is a palindrome, considering only alphanumeric characters and ignoring cases.

For example,\\
\code{"A man, a plan, a canal: Panama"} is a palindrome.\\
\code{"race a car"} is not a palindrome.

Note:
Have you consider that the string might be empty? This is a good question to ask during an interview.

For the purpose of this problem, we define empty string as valid palindrome.


\subsubsection{分析}
无


\subsubsection{代码}
\begin{Code}
// Leet Code, Valid Palindrome
class Solution {
public:
    bool isPalindrome(string s) {
        transform(s.begin(), s.end(), s.begin(), ::tolower);
        auto left = s.begin(), right = prev(s.end());
        while (left < right) {
            if (!::isalnum(*left)) {
                ++left;
                continue;
            }
            if (!::isalnum(*right)) {
                --right;
                continue;
            }
            if (*left == *right) {
                ++left;
                --right;
            } else {
                return false;
            }
        }
        return true;
    }
};
\end{Code}


\subsubsection{相关题目}
\begindot
\item 无
\myenddot


\section{Implement strStr()} %%%%%%%%%%%%%%%%%%%%%%%%%%%%%%
\label{sec:strstr}


\subsubsection{描述}
Implement strStr().

Returns a pointer to the first occurrence of needle in haystack, or null if needle is not part of haystack.


\subsubsection{分析}
暴力算法的复杂度是 $O(m*n)$,代码如下。更高效的的算法有KMP算法、Boyer-Mooer算法和Rabin-Karp算法。面试中暴力算法足够了,一定要写得没有BUG。


\subsubsection{代码}
\begin{Code}
// LeetCode, Implement strStr()
// 暴力解法,复杂度O(N*M)
class Solution {
public:
    char *strStr(const char *haystack, const char *needle) {
        // if needle is empty return the full string
        if (!*needle) return (char*) haystack;

        const char *p1;
        const char *p2;
        const char *p1_advance = haystack;
        for (p2 = &needle[1]; *p2; ++p2) {
            p1_advance++;   // advance p1_advance M-1 times
        }

        for (p1 = haystack; *p1_advance; p1_advance++) {
            char *p1_old = (char*) p1;
            p2 = needle;
            while (*p1 && *p2 && *p1 == *p2) {
                p1++;
                p2++;
            }
            if (!*p2) return p1_old;

            p1 = p1_old + 1;
        }
        return NULL;
    }
};
\end{Code}


\subsubsection{相关题目}
\begindot
\item String to Integer (atoi) ,见 \S \ref{sec:string-to-integer}
\myenddot


\section{String to Integer (atoi)} %%%%%%%%%%%%%%%%%%%%%%%%%%%%%%
\label{sec:string-to-integer}


\subsubsection{描述}
Implement \fn{atoi} to convert a string to an integer.

\textbf{Hint}: Carefully consider all possible input cases. If you want a challenge, please do not see below and ask yourself what are the possible input cases.

\textbf{Notes}: It is intended for this problem to be specified vaguely (ie, no given input specs). You are responsible to gather all the input requirements up front.

\textbf{Requirements for atoi}:

The function first discards as many whitespace characters as necessary until the first non-whitespace character is found. Then, starting from this character, takes an optional initial plus or minus sign followed by as many numerical digits as possible, and interprets them as a numerical value.

The string can contain additional characters after those that form the integral number, which are ignored and have no effect on the behavior of this function.

If the first sequence of non-whitespace characters in str is not a valid integral number, or if no such sequence exists because either str is empty or it contains only whitespace characters, no conversion is performed.

If no valid conversion could be performed, a zero value is returned. If the correct value is out of the range of representable values, \code{INT_MAX (2147483647)} or \code{INT_MIN (-2147483648)} is returned.

\subsubsection{分析}
细节题。注意几个测试用例:
\begin{enumerate}
\item 不规则输入,但是有效,"-3924x8fc", "  +  413",
\item 无效格式," ++c", " ++1"
\item 溢出数据,"2147483648"
\end{enumerate}

\subsubsection{代码}
\begin{Code}
// LeetCode, String to Integer (atoi)
class Solution {
public:
    int atoi(const char *str) {
        int num = 0;
        int sign = 1;
        const int len = strlen(str);
        int i = 0;

        while (str[i] == ' ' && i < len) i++;

        if (str[i] == '+') i++;

        if (str[i] == '-') {
            sign = -1;
            i++;
        }

        for (; i < len; i++) {
            if (str[i] < '0' || str[i] > '9')
                break;
            if (num > INT_MAX / 10 ||
                            (num == INT_MAX / 10 &&
                                    (str[i] - '0') > INT_MAX % 10)) {
                return sign == -1 ? INT_MIN : INT_MAX;
            }
            num = num * 10 + str[i] - '0';
        }
        return num * sign;
    }
};
\end{Code}


\subsubsection{相关题目}
\begindot
\item Implement strStr() ,见 \S \ref{sec:strstr}
\myenddot


\section{Add Binary} %%%%%%%%%%%%%%%%%%%%%%%%%%%%%%
\label{sec:add-binary}


\subsubsection{描述}
Given two binary strings, return their sum (also a binary string).

For example,
\begin{Code}
a = "11"
b = "1"
\end{Code}
Return {\small \fontspec{Latin Modern Mono} "100"}.


\subsubsection{分析}
无


\subsubsection{代码}
\begin{Code}
//LeetCode, Add Binary
class Solution {
public:
    string addBinary(string a, string b) {
        string result;
        const size_t max_len = a.size() > b.size() ? a.size() : b.size();
        reverse(a.begin(), a.end());
        reverse(b.begin(), b.end());
        int carry = 0;
        for (size_t i = 0; i < max_len; i++) {
            const int ai = i < a.size() ? a[i] - '0' : 0;
            const int bi = i < b.size() ? b[i] - '0' : 0;
            const int val = (ai + bi + carry) % 2;
            carry = (ai + bi + carry) / 2;
            result.insert(result.begin(), val + '0');
        }
        if (carry == 1) {
            result.insert(result.begin(), '1');
        }
        return result;
    }
};
\end{Code}


\subsubsection{相关题目}
\begindot
\item Add Two Numbers, 见 \S \ref{sec:add-two-numbers}
\myenddot


\section{Longest Palindromic Substring} %%%%%%%%%%%%%%%%%%%%%%%%%%%%%%
\label{sec:longest-palindromic-substring}


\subsubsection{描述}
Given a string $S$, find the longest palindromic substring in $S$. You may assume that the maximum length of $S$ is 1000, and there exists one unique longest palindromic substring.


\subsubsection{分析}
最长回文子串,非常经典的题。

思路一:暴力枚举,以每个元素为中间元素,同时从左右出发,复杂度$O(n^2)$。

思路二:记忆化搜索,复杂度$O(n^2)$。设\fn{f[i][j]} 表示[i,j]之间的最长回文子串,递推方程如下:
\begin{Code}
f[i][j] = if (i == j) S[i]
          if (S[i] == S[j] && f[i+1][j-1] == S[i+1][j-1]) S[i][j]
          else max(f[i+1][j-1], f[i][j-1], f[i+1][j])
\end{Code}

思路三:动规,复杂度$O(n^2)$。设状态为\fn{f(i,j)},表示区间[i,j]是否为回文串,则状态转移方程为
$$
f(i,j)=\begin{cases}
true & ,i=j\\
S[i]=S[j] & , j = i + 1 \\
S[i]=S[j] \text{ and } f(i+1, j-1) & , j > i + 1
\end{cases}
$$

思路三:Manacher’s Algorithm, 复杂度$O(n)$。详细解释见 \myurl{http://leetcode.com/2011/11/longest-palindromic-substring-part-ii.html} 。


\subsubsection{代码}

\begin{Code}
// LeetCode, Longest Palindromic Substring
// 备忘录法,TLE
typedef string::const_iterator Iterator;

namespace std {
template<>
struct hash<pair<Iterator, Iterator>> {
    size_t operator()(pair<Iterator, Iterator> const& p) const {
        return ((size_t) &(*p.first)) ^ ((size_t) &(*p.second));
    }
};
}

class Solution {
public:
    string longestPalindrome(string const& s) {
        cache.clear();
        return cachedLongestPalindrome(s.begin(), s.end());
    }

private:
    unordered_map<pair<Iterator, Iterator>, string> cache;

    string longestPalindrome(Iterator first, Iterator last) {
        size_t length = distance(first, last);

        if (length < 2) return string(first, last);

        auto s = cachedLongestPalindrome(next(first), prev(last));

        if (s.length() == length - 2 && *first == *prev(last))
            return string(first, last);

        auto s1 = cachedLongestPalindrome(next(first), last);
        auto s2 = cachedLongestPalindrome(first, prev(last));

        // return max(s, s1, s2)
        if (s.size() > s1.size()) return s.size() > s2.size() ? s : s2;
        else return s1.size() > s2.size() ? s1 : s2;
    }

    string cachedLongestPalindrome(Iterator first, Iterator last) {
        auto key = make_pair(first, last);
        auto pos = cache.find(key);

        if (pos != cache.end()) return pos->second;
        else return cache[key] = longestPalindrome(first, last);
    }
};
\end{Code}

\begin{Code}
// LeetCode, Longest Palindromic Substring
// 动规
class Solution {
public:
    string longestPalindrome(string s) {
        const int len = s.size();
        int f[len][len];
        memset(f, 0, len * len * sizeof(int)); //TODO: fill, fill_n
        int maxL = 1, start = 0;  // 最长回文子串的长度,起点

        for (size_t i = 0; i < s.size(); i++) {
            f[i][i] = 1;
            for (size_t j = 0; j < i; j++) {  // [j, i]
                f[j][i] = (s[j] == s[i] && (i - j < 2 || f[j + 1][i - 1]));
                if (f[j][i] && maxL < (i - j + 1)) {
                    maxL = i - j + 1;
                    start = j;
                }
            }
        }
        return s.substr(start, maxL);
    }
};
\end{Code}

\begin{Code}
// LeetCode, Longest Palindromic Substring
// Manacher’s Algorithm
class Solution {
public:
    // Transform S into T.
    // For example, S = "abba", T = "^#a#b#b#a#$".
    // ^ and $ signs are sentinels appended to each end to avoid bounds checking
    string preProcess(string s) {
        int n = s.length();
        if (n == 0) return "^$";

        string ret = "^";
        for (int i = 0; i < n; i++) ret += "#" + s.substr(i, 1);

        ret += "#$";
        return ret;
    }

    string longestPalindrome(string s) {
        string T = preProcess(s);
        int n = T.length();
        // 以T[i]为中心,向左/右扩张的长度,不包含T[i]自己,
        // 因此 P[i]是源字符串中回文串的长度
        int *P = new int[n];
        int C = 0, R = 0;

        for (int i = 1; i < n - 1; i++) {
            int i_mirror = 2 * C - i; // equals to i' = C - (i-C)

            P[i] = (R > i) ? min(R - i, P[i_mirror]) : 0;

            // Attempt to expand palindrome centered at i
            while (T[i + 1 + P[i]] == T[i - 1 - P[i]])
                P[i]++;

            // If palindrome centered at i expand past R,
            // adjust center based on expanded palindrome.
            if (i + P[i] > R) {
                C = i;
                R = i + P[i];
            }
        }

        // Find the maximum element in P.
        int maxLen = 0;
        int centerIndex = 0;
        for (int i = 1; i < n - 1; i++) {
            if (P[i] > maxLen) {
                maxLen = P[i];
                centerIndex = i;
            }
        }
        delete[] P;

        return s.substr((centerIndex - 1 - maxLen) / 2, maxLen);
    }
};
\end{Code}


\subsubsection{相关题目}
\begindot
\item 无
\myenddot


\section{Regular Expression Matching} %%%%%%%%%%%%%%%%%%%%%%%%%%%%%%
\label{sec:regular-expression-matching}


\subsubsection{描述}
Implement regular expression matching with support for \fn{'.'} and \fn{'*'}.

\fn{'.'} Matches any single character.
\fn{'*'} Matches zero or more of the preceding element.

The matching should cover the entire input string (not partial).

The function prototype should be:
\begin{Code}
bool isMatch(const char *s, const char *p)
\end{Code}

Some examples:
\begin{Code}
isMatch("aa","a") → false
isMatch("aa","aa") → true
isMatch("aaa","aa") → false
isMatch("aa", "a*") → true
isMatch("aa", ".*") → true
isMatch("ab", ".*") → true
isMatch("aab", "c*a*b") → true
\end{Code}


\subsubsection{分析}
这是一道很有挑战的题。


\subsubsection{代码}
递归版。
\begin{Code}
// LeetCode, Regular Expression Matching
// 递归版
class Solution {
public:
    bool isMatch(const char *s, const char *p) {
        if (*p == '\0') return *s == '\0';

        if (*(p + 1) != '*') {
            if (*p == *s || (*p == '.' && *s != '\0'))
                return isMatch(s + 1, p + 1);
            else
                return false;
        } else {
            while (*p == *s || (*p == '.' && *s != '\0')) {
                if (isMatch(s, p + 2))
                    return true;
                s++;
            }
            return isMatch(s, p + 2);
        }
    }
};
\end{Code}

迭代版。
\begin{Code}

\end{Code}


\subsubsection{相关题目}
\begindot
\item Wildcard Matching, 见 \S \ref{sec:wildcard-matching}
\myenddot


\section{Wildcard Matching} %%%%%%%%%%%%%%%%%%%%%%%%%%%%%%
\label{sec:wildcard-matching}


\subsubsection{描述}
Implement wildcard pattern matching with support for \fn{'?'} and \fn{'*'}.

\fn{'?'} Matches any single character.
\fn{'*'} Matches any sequence of characters (including the empty sequence).

The matching should cover the entire input string (not partial).

The function prototype should be:
\begin{Code}
bool isMatch(const char *s, const char *p)
\end{Code}

Some examples:
\begin{Code}
isMatch("aa","a") → false
isMatch("aa","aa") → true
isMatch("aaa","aa") → false
isMatch("aa", "*") → true
isMatch("aa", "a*") → true
isMatch("ab", "?*") → true
isMatch("aab", "c*a*b") → false
\end{Code}


\subsubsection{分析}
跟上一题很类似。

主要是\fn{'*'}的匹配问题。\fn{p}每遇到一个\fn{'*'},就保留住当前\fn{'*'}的坐标和\fn{s}的坐标,然后\fn{s}从前往后扫描,如果不成功,则\fn{s++},重新扫描。


\subsubsection{代码}
递归版。
\begin{Code}

\end{Code}

迭代版。
\begin{Code}
// LeetCode, Wildcard Matching
// 迭代版
class Solution {
public:
    bool isMatch(const char *s, const char *p) {
        bool star = false;
        const char *str, *ptr;
        for (str = s, ptr = p; *str != '\0'; str++, ptr++) {
            switch (*ptr) {
            case '?':
                break;
            case '*':
                star = true;
                s = str, p = ptr;
                while (*p == '*') p++;
                if (*p == '\0') return true;
                str = s - 1;
                ptr = p - 1;
                break;
            default:
                if (*str != *ptr) {
                    // 如果前面没有'*',则匹配不成功
                    if (!star) return false;
                    s++;
                    str = s - 1;
                    ptr = p - 1;
                }
            }
        }
        while (*ptr == '*') ptr++;
        return (*ptr == '\0');
    }
};
\end{Code}


\subsubsection{相关题目}
\begindot
\item Regular Expression Matching, 见 \S \ref{sec:regular-expression-matching}
\myenddot


\section{Longest Common Prefix} %%%%%%%%%%%%%%%%%%%%%%%%%%%%%%
\label{sec:longest-common-prefix}


\subsubsection{描述}
Write a function to find the longest common prefix string amongst an array of strings.


\subsubsection{分析}
从位置0开始,对每一个位置比较所有字符串,直到遇到一个不匹配。


\subsubsection{代码}
\begin{Code}
// LeetCode, Longest Common Prefix
// 从位置0开始,对每一个位置比较所有字符串,直到遇到一个不匹配
class Solution {
public:
    string longestCommonPrefix(vector<string> &strs) {
        if (strs.size() == 0) return "";
        string prefix;
        int pos = 0; // 当前位置,等于 prefix 的长度

        while (true) {
            char c;
            int i = 0;
            for (; i < strs.size(); i++) {
                if (i == 0) c = strs[0][pos];
                if (strs[i].size() == pos || c != strs[i][pos])
                    break;
            }
            if (i != strs.size()) break;
            ++pos;
            prefix.append(1, c);
        }
        return prefix;
    }
};
\end{Code}


\subsubsection{相关题目}
\begindot
\item 无
\myenddot


\section{Valid Number} %%%%%%%%%%%%%%%%%%%%%%%%%%%%%%
\label{sec:valid-number}


\subsubsection{描述}
Validate if a given string is numeric.

Some examples:
\begin{Code}
"0" => true
" 0.1 " => true
"abc" => false
"1 a" => false
"2e10" => true
\end{Code}

Note: It is intended for the problem statement to be ambiguous. You should gather all requirements up front before implementing one.


\subsubsection{分析}
细节实现题。

本题的功能与标准库中的\fn{strtod()}功能类似。


\subsubsection{代码}
\begin{Code}
// LeetCode, Valid Number
// @author 龚陆安 (http://weibo.com/luangong)
// finite automata
class Solution {
public:
    bool isNumber(const char *s) {
        enum InputType {
            INVALID,    // 0
            SPACE,      // 1
            SIGN,       // 2
            DIGIT,      // 3
            DOT,        // 4
            EXPONENT,   // 5
            NUM_INPUTS  // 6
        };
        const int transitionTable[][NUM_INPUTS] = {
                -1, 0, 3, 1, 2, -1, // next states for state 0
                -1, 8, -1, 1, 4, 5,     // next states for state 1
                -1, -1, -1, 4, -1, -1,     // next states for state 2
                -1, -1, -1, 1, 2, -1,     // next states for state 3
                -1, 8, -1, 4, -1, 5,     // next states for state 4
                -1, -1, 6, 7, -1, -1,     // next states for state 5
                -1, -1, -1, 7, -1, -1,     // next states for state 6
                -1, 8, -1, 7, -1, -1,     // next states for state 7
                -1, 8, -1, -1, -1, -1,     // next states for state 8
                };

        int state = 0;
        for (; *s != '\0'; ++s) {
            InputType inputType = INVALID;
            if (isspace(*s))
                inputType = SPACE;
            else if (*s == '+' || *s == '-')
                inputType = SIGN;
            else if (isdigit(*s))
                inputType = DIGIT;
            else if (*s == '.')
                inputType = DOT;
            else if (*s == 'e' || *s == 'E')
                inputType = EXPONENT;

            // Get next state from current state and input symbol
            state = transitionTable[state][inputType];

            // Invalid input
            if (state == -1) return false;
        }
        // If the current state belongs to one of the accepting (final) states,
        // then the number is valid
        return state == 1 || state == 4 || state == 7 || state == 8;

    }
};
\end{Code}

\begin{Code}
// LeetCode, Valid Number
// @author 连城 (http://weibo.com/lianchengzju)
// 偷懒,直接用 strtod()
class Solution {
public:
    bool isNumber (char const* s) {
        char* endptr;
        strtod (s, &endptr);

        if (endptr == s) return false;

        for (; *endptr; ++endptr)
            if (!isspace (*endptr)) return false;

        return true;
    }
};
\end{Code}


\subsubsection{相关题目}
\begindot
\item 无
\myenddot


\section{Integer to Roman} %%%%%%%%%%%%%%%%%%%%%%%%%%%%%%
\label{sec:integer-to-roman}


\subsubsection{描述}
Given an integer, convert it to a roman numeral.

Input is guaranteed to be within the range from 1 to 3999.


\subsubsection{分析}
无


\subsubsection{代码}
\begin{Code}
// LeetCode, Integer to Roman
// @author 连城 (http://weibo.com/lianchengzju)
class Solution {
public:
    string intToRoman (int n) {
        vector<pair<int, string>> radixes {
            { 1000, "M" },
            { 900, "CM" },
            { 500, "D" },
            { 400, "CD" },
            { 100, "C" },
            { 90, "XC" },
            { 50, "L" },
            { 40, "XL" },
            { 10, "X" },
            { 9, "IX" },
            { 5, "V" },
            { 4, "IV" },
            { 1, "I" }
        };

        string roman;

        while (n > 0) {
            auto radix = radixes.begin();

            for (; n / radix->first == 0; ++radix)
                ;

            for (int i = 0; i < n / radix->first; ++i)
                roman += radix->second;

            n %= radix->first;
        }

        return roman;
    }
};
\end{Code}


\subsubsection{相关题目}
\begindot
\item Roman to Integer, 见 \S \ref{sec:roman-to-integer}
\myenddot


\section{Roman to Integer} %%%%%%%%%%%%%%%%%%%%%%%%%%%%%%
\label{sec:roman-to-integer}


\subsubsection{描述}
Given a roman numeral, convert it to an integer.

Input is guaranteed to be within the range from 1 to 3999.


\subsubsection{分析}
从前往后扫描,用一个临时变量记录分段数字。

如果当前比前一个大,说明这一段的值应该是当前这个值减去上一个值。比如\fn{IV = 5 – 1};否则,将当前值加入到结果中,然后开始下一段记录。比如\fn{VI = 5 + 1, II=1+1}


\subsubsection{代码}
\begin{Code}
// LeetCode, Roman to Integer
class Solution {
public:
    inline int c2n(const char c) {
        switch (c) {
        case 'I': return 1;
        case 'V': return 5;
        case 'X': return 10;
        case 'L': return 50;
        case 'C': return 100;
        case 'D': return 500;
        case 'M': return 1000;
        default: return 0;
        }
    }

    int romanToInt(string s) {
        int result = 0;
        for (size_t i = 0; i < s.size(); i++) {
            if (i > 0 && c2n(s[i]) > c2n(s[i - 1])) {
                result += (c2n(s[i]) - 2 * c2n(s[i - 1]));
            } else {
                result += c2n(s[i]);
            }
        }
        return result;
    }
};
\end{Code}


\subsubsection{相关题目}
\begindot
\item Integer to Roman, 见 \S \ref{sec:integer-to-roman}
\myenddot


\section{Count and Say} %%%%%%%%%%%%%%%%%%%%%%%%%%%%%%
\label{sec:count-and-say}


\subsubsection{描述}
The count-and-say sequence is the sequence of integers beginning as follows:
\begin{Code}
1, 11, 21, 1211, 111221, ...
\end{Code}

\fn{1} is read off as \fn{"one 1"} or \fn{11}.

\fn{11} is read off as \fn{"two 1s"} or \fn{21}.

\fn{21} is read off as \fn{"one 2"}, then \fn{"one 1"} or \fn{1211}.

Given an integer $n$, generate the nth sequence.

Note: The sequence of integers will be represented as a string.


\subsubsection{分析}
模拟。


\subsubsection{代码}
\begin{Code}
// LeetCode, Count and Say
// @author 连城 (http://weibo.com/lianchengzju)
class Solution {
public:
    string countAndSay(int n) {
        string s("1");

        while (--n)
            s = get_next(s);

        return s;
    }

    string get_next(const string &s) {
        stringstream ss;

        for (auto i = s.begin(); i != s.end(); ) {
            auto j = find_if(i, s.end(), bind1st(not_equal_to<char>(), *i));
            ss << distance(i, j) << *i;
            i = j;
        }

        return ss.str();
    }
};
\end{Code}


\subsubsection{相关题目}
\begindot
\item 无
\myenddot


\section{Anagrams} %%%%%%%%%%%%%%%%%%%%%%%%%%%%%%
\label{sec:anagrams}


\subsubsection{描述}
Given an array of strings, return all groups of strings that are anagrams.

Note: All inputs will be in lower-case.


\subsubsection{分析}
Anagram(回文构词法)是指打乱字母顺序从而得到新的单词,比如 \fn{"dormitory"} 打乱字母顺序会变成 \fn{"dirty room"} ,\fn{"tea"} 会变成\fn{"eat"}。

回文构词法有一个特点:单词里的字母的种类和数目没有改变,只是改变了字母的排列顺序。因此,将几个单词按照字母顺序排序后,若它们相等,则它们属于同一组 anagrams 。


\subsubsection{代码}
\begin{Code}
// LeetCode, Anagrams
class Solution {
public:
    vector<string> anagrams(vector<string> &strs) {
        unordered_map<string, vector<string> > group;
        for (const auto &s : strs) {
            string key = s;
            sort(key.begin(), key.end());
            group[key].push_back(s);
        }

        vector<string> result;
        for (auto it = group.begin(); it != group.end(); it++) {
            if (it->second.size() > 1)
                result.insert(result.end(), it->second.begin(), it->second.end());
        }
        return result;
    }
};
\end{Code}


\subsubsection{相关题目}
\begindot
\item 无
\myenddot


\section{Simplify Path} %%%%%%%%%%%%%%%%%%%%%%%%%%%%%%
\label{sec:simplify-path}


\subsubsection{描述}
Given an absolute path for a file (Unix-style), simplify it.

For example, \\
path = \fn{"/home/"}, => \fn{"/home"} \\
path = \fn{"/a/./b/../../c/"}, => \fn{"/c"} \\

Corner Cases:
\begindot
\item Did you consider the case where path = \fn{"/../"}? 
In this case, you should return \fn{"/"}.
\item 
Another corner case is the path might contain multiple slashes \fn{'/'} together, such as \fn{"/home//foo/"}.
In this case, you should ignore redundant slashes and return \fn{"/home/foo"}.
\myenddot


\subsubsection{分析}
很有实际价值的题目。


\subsubsection{代码}
\begin{Code}
// LeetCode, Simplify Path
class Solution {
public:
    string simplifyPath(string const& path) {
        vector<string> dirs; // 当做栈

        for (auto i = path.begin(); i != path.end();) {
            ++i;

            auto j = find(i, path.end(), '/');
            auto dir = string(i, j);

            if (!dir.empty() && dir != ".") {// 当有连续 '///'时,dir 为空
                if (dir == "..") {
                    if (!dirs.empty())
                        dirs.pop_back();
                } else
                    dirs.push_back(dir);
            }

            i = j;
        }

        stringstream out;
        if (dirs.empty()) {
            out << "/";
        } else {
            for (auto dir : dirs)
                out << '/' << dir;
        }

        return out.str();
    }
};
\end{Code}


\subsubsection{相关题目}
\begindot
\item 无
\myenddot


\section{Length of Last Word} %%%%%%%%%%%%%%%%%%%%%%%%%%%%%%
\label{sec:length-of-last-word}


\subsubsection{描述}
Given a string s consists of upper/lower-case alphabets and empty space characters \fn{' '}, return the length of last word in the string.

If the last word does not exist, return 0.

Note: A word is defined as a character sequence consists of non-space characters only.

For example, 
Given \fn{s = "Hello World"},
return 5.


\subsubsection{分析}
细节实现题。


\subsubsection{代码}
\begin{Code}
// LeetCode, Length of Last Word
// 偷懒,用 STL
class Solution {
public:
    int lengthOfLastWord(const char *s) {
        const string str(s);
        auto first = find_if(str.rbegin(), str.rend(), ::isalpha);
        auto last = find_if_not(first, str.rend(), ::isalpha);
        return distance(first, last);
    }
};
\end{Code}


\begin{Code}
// LeetCode, Length of Last Word
// 顺序扫描,记录每个 word 的长度
class Solution {
public:
    int lengthOfLastWord(const char *s) {
        int len = 0;
        while (*s) {
            if (*(s++) != ' ')
                ++len;
            else if (*s && *s != ' ')
                len = 0;
        }
        return len;
    }
};
\end{Code}


\subsubsection{相关题目}
\begindot
\item 无
\myenddot
